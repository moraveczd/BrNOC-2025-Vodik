\documentclass[hyperref=unicode,presentation,10pt]{beamer}

\usepackage[absolute,overlay]{textpos}
\usepackage{array}
\usepackage{graphicx}
\usepackage{adjustbox}
\usepackage[version=4]{mhchem}
\usepackage{chemfig}
\usepackage{caption}

%dělení slov
\usepackage{ragged2e}
\let\raggedright=\RaggedRight
%konec dělení slov

\addtobeamertemplate{frametitle}{
	\let\insertframetitle\insertsectionhead}{}
\addtobeamertemplate{frametitle}{
	\let\insertframesubtitle\insertsubsectionhead}{}

\makeatletter
\CheckCommand*\beamer@checkframetitle{\@ifnextchar\bgroup\beamer@inlineframetitle{}}
\renewcommand*\beamer@checkframetitle{\global\let\beamer@frametitle\relax\@ifnextchar\bgroup\beamer@inlineframetitle{}}
\makeatother
\setbeamercolor{section in toc}{fg=red}
\setbeamertemplate{section in toc shaded}[default][100]

\usepackage{fontspec}
\usepackage{unicode-math}

\usepackage{polyglossia}
\setdefaultlanguage{czech}

\def\uv#1{„#1“}

\mode<presentation>{\usetheme{default}}
 \usecolortheme{beaver}

\setbeamertemplate{footline}[frame number]

\title[Crisis]
{Vodíkové hospodářství a chemie}

\subtitle{BrNOC 2025}
%\author{\href{http://z-moravec.net/chemie/zaklady-chemie/}{http://z-moravec.net/}}
\author{Zdeněk Moravec, hugo@chemi.muni.cz \\
\adjincludegraphics[height=50mm]{img/IUPAC_PSP.jpg}}
\date{}

\begin{document}

\begin{frame}
	\titlepage
\end{frame}

\section{Úvod}
\frame{
	\frametitle{}
	\vfill
	\begin{columns}
		\begin{column}{.38\textwidth}
			\textbf{Osnova}
			\begin{enumerate}
				\item Vodík
				\item Vodíkové hospodářství
				\item Výroba vodíku
				\item Skladování vodíku
				\item Využití vodíku
			\end{enumerate}
		\end{column}
		\begin{column}{.65\textwidth}
			\begin{figure}
				\adjincludegraphics[width=\textwidth]{img/Hydrogen_balloon_explosion.jpg}
				\caption*{Výbuch vodíku.\footnote[frame]{Zdroj: \href{https://commons.wikimedia.org/wiki/File:Hydrogen_balloon_explosion.jpg}{Maxim Bilovitskiy/Commons}}}
			\end{figure}
		\end{column}
	\end{columns}
	\vfill
}

\section{Vodík}
\frame{
	\frametitle{}
	\vfill
	\begin{itemize}
		\item Nejjednodušší prvek. V přírodě se vyskytuje ve formě tří izotopů:\footnote[frame]{\href{https://www2.lbl.gov/abc/wallchart/chapters/02/3.html}{The Isotopes of Hydrogen}}
		\begin{itemize}
			\item \ce{^1_1H} -- protium, nejběžnější, zastoupení v přírodě 99,9885~\%.
			\item \ce{^2_1H \bond{3} D} -- deuterium, zastoupení v přírodě 0,0115~\%.
			\item \ce{^3_1H \bond{3} T} -- tritium, nestabilní izotop, $T_\frac{1}{2}$ = 12,32~let.
			\item \ce{^3_1H ->[12,32 let] ^3_2He + $\beta$^-}
		\end{itemize}
		\item Plynný prvek, vytváří dvouatomové molekuly \ce{H2}.
		\item Je to nejrozšířenější prvek ve vesmíru.
		\item Díky své nízké hustotě se dříve využíval ve vzducholodích, nyní se už využívá jen v meteorologických a pouťových balónech.
		\item Dnes se využívá jako redukční činidlo v organické syntéze a metalurgii.\footnote[frame]{\href{https://doi.org/10.1021/j150584a002}{Nascent Hydrogen}}
		\item Je také výchozí látkou při Haberově--Boschově syntéze amoniaku.\footnote[frame]{\href{https://www.thechemicalengineer.com/features/cewctw-fritz-haber-and-carl-bosch-feed-the-world/}{Fritz Haber and Carl Bosch – Feed the World}}
		\item Se vzduchem tvoří výbušnou směs.\footnote[frame]{\href{https://www.youtube.com/watch?v=XlmtcsSJr7Q}{Hydrogen/Oxygen Balloon Explosions}}
	\end{itemize}
	\vfill
}

\frame{
	\frametitle{}
	\vfill
	\begin{figure}
		\adjincludegraphics[width=.8\textwidth]{img/Hindenburg_burning_1937.jpg}
		\caption*{Výbuch vzducholodi Hindenburg, 1937. 36 mrtvých, 62 zraněných osob.\footnote[frame]{Zdroj: \href{https://commons.wikimedia.org/wiki/File:Hindenburg_disaster.jpg}{Sam Sphere/Commons}}}
	\end{figure}
	\vfill
}

\subsection{Spinové izomery divodíku}
\frame{
	\frametitle{}
	\vfill
	\begin{columns}
		\begin{column}{.6\textwidth}
			\begin{itemize}
				\item U molekul \ce{H2} byla zjištěna existence \textit{spinových izomerů}, ty se liší vzájemnou orientací \textit{jaderných spinů}.\footnote[frame]{\href{https://chem.libretexts.org/Bookshelves/Physical_and_Theoretical_Chemistry_Textbook_Maps/Supplemental_Modules_(Physical_and_Theoretical_Chemistry)/Quantum_Mechanics/11\%3A_Molecules/Ortho_and_Para_hydrogen}{Ortho and Para hydrogen}}
				\item Pokud mají obě jádra spiny orientovány paralelně, jde o \textit{ortho}-vodík, v~opačném případě jde o \textit{para}-vodík.
				\item Za laboratorní teploty je obsah \textit{ortho}-izomeru zhruba 75~\%.
				\item Snižováním teploty stoupá koncentrace \textit{para}-izomeru.
				\item Při nízkých teplotách lze získat čistý \textit{para}-vodík, ale zahříváním nelze získat vyšší koncentraci \textit{ortho}-izomeru než 75~\%.
			\end{itemize}
		\end{column}
		\begin{column}{.4\textwidth}
			\begin{figure}
				\adjincludegraphics[width=\textwidth]{img/Spinisomers_of_molecular_hydrogen.png}
				\caption*{Spinové izomery divodíku.\footnote[frame]{Zdroj: \href{https://commons.wikimedia.org/wiki/File:Spinisomers_of_molecular_hydrogen.png}{Xaa/Commons}}}
			\end{figure}
		\end{column}
	\end{columns}
	\vfill
}

\subsection{Vodíková vazba}
\frame{
	\frametitle{}
	\vfill
	\begin{columns}
		\begin{column}{.6\textwidth}
			\begin{itemize}
				\item Intermolekulární i intramolekulární vazba.
				\item Pokud je atom vodíku vázán k elektronegativnímu atomu (F, N, O), dojde k velkému snížení elektronové hustoty v jeho okolí a tím i k zesílení jeho interakce se zápornými náboji v systému.
				\item Vodíková vazba je odpovědná za vysokou teplotu vody a také za strukturu ledu.
				\item Má velmi důležitou úlohu v biochemii, setkáme se s ní např. v DNA, proteinech, atd.
			\end{itemize}
		\end{column}
		\begin{column}{.4\textwidth}
			\begin{figure}
				\adjincludegraphics[width=\textwidth]{img/Wasserstoffbrückenbindungen-Wasser.png}
				\caption*{Vodíková vazba ve vodě.\footnote[frame]{Zdroj: \href{https://commons.wikimedia.org/wiki/File:Wasserstoffbrückenbindungen-Wasser.svg}{Raimund Apfelbach/Commons}}}
			\end{figure}
		\end{column}
	\end{columns}
	\vfill
}

\subsection{Redukční činidlo}
\frame{
	\frametitle{}
	\vfill
	\begin{itemize}
		\item Vodík se využívá jako účinné redukční činidlo.
	\end{itemize}
	\begin{figure}
		\adjincludegraphics[width=0.8\textwidth]{img/BiRedukce.jpg}
		\caption*{Redukce oxidu bismutitého vodíkem.}
	\end{figure}
	\vfill
}


\frame{
	\frametitle{}
	\vfill
	\begin{itemize}
		\item Nejvyšší účinnosti se dosahuje s využitím tzv. \textit{nascentního vodíku}.
		\item Mezi účinné redukční činidla patří i komplexní hydridy, např. \ce{NaBH4} nebo	tzv. \textit{superhydrid}, triethylborohydrid lithný.\footnote[frame]{\href{https://www.organic-chemistry.org/chemicals/reductions/lithiumtriethylborohydride.shtm}{Lithium triethylborohydride, LiTEBH, Superhydride}}
		\item \ce{LiH + Et3B -> LiEt3BH}
	\end{itemize}
	\begin{figure}
		\adjincludegraphics[width=.8\textwidth]{img/Lithium_triethylborohydride.png}
	\end{figure}
	\vfill
}

\frame{
	\frametitle{}
	\textbf{Raneyův nikl}
	\begin{itemize}
		\item Šedý prášek, složený převážně z niklu, vyrábí se ze slitiny niklu s hliníkem.\footnote[frame]{\href{https://www.masterorganicchemistry.com/2011/09/30/reagent-friday-raney-nickel/}{Reagent Friday: Raney Nickel}}
		\item Ta je rozpuštěna v hydroxidu sodném, hliník se rozpustí za vývoje vodíku, který se nasorbuje na povrch zrn niklu.
		\item \ce{2 Al + 2 NaOH + 6 H2O -> 2 Na[Al(OH)4] + 3 H2}
		\item Díky obsahu vodíku je Raneyův nikl pyroforický a musí se uchovávat tak, aby se zabránilo kontaktu se vzdušným kyslíkem.\footnote[frame]{\href{https://www.youtube.com/watch?v=KEOcufv1qdM}{Raney Nickel spontaneous combustion}}
	\end{itemize}
	\begin{figure}
		\adjincludegraphics[height=.35\textheight]{img/Dry_Raney_nickel.jpg}
	\end{figure}
}

\subsection{Jaderná fůze}
\frame{
	\frametitle{}
	\vfill
	\begin{columns}
		\begin{column}{.75\textwidth}
			\begin{itemize}
				\item Jedno z možných využití vodíku v energetice, resp. deuteria a tritia je \textit{jaderná fůze}.
				\item Místo štěpení těžkých jader, dochází ke slučování lehkých jader.
				\item Nejvýhodnější je využití izotopů vodíku, které se slučují za vzniku jader helia a uvolnění energie.
				\begin{itemize}
					\item \ce{^2_1H + ^3_1H -> ^4_2He + ^1_0n}
				\end{itemize}
				\item Aby ke slučování mohlo dojít, je nutné v místě reakce vytvořit plasma o teplotě 100 miliónů $^\circ$C.\footnote[frame]{\href{https://www.aldebaran.cz/zvuky/blyskani/docs/10.html}{Zapálíme Slunce na Zemi?}}
				\item Gram směsi deuteria s tritiem by měl být schopen generovat výkon 500 MW po dobu asi jedné minuty.
			\end{itemize}
		\end{column}
		\begin{column}{.4\textwidth}
			\begin{figure}
				\adjincludegraphics[height=40mm]{img/Deuterium-tritium_fusion_-_comma.png}
				\caption*{Jaderná fúze deuteria s tritiem.\footnote[frame]{Zdroj: \href{https://commons.wikimedia.org/wiki/File:Deuterium-tritium_fusion_-_comma.svg}{Wykis/Commons}}}
			\end{figure}
		\end{column}
	\end{columns}
	\vfill
}

\frame{
	\frametitle{}
	\vfill
	\begin{columns}
		\begin{column}{.7\textwidth}
			\begin{itemize}
				\item \textit{ITER} -- International Thermonuclear Experimental Reactor.\footnote[frame]{\href{https://www.iter.org/}{ITER}}
				\item Stavba probíhá na francouzském území, začala v roce 2007.
				\item Mezinárodní projekt, jehož cílem je konstrukce reaktoru, který bude schopen vyrábět elektřinu pomocí jaderné fúze.
				\item Aby bylo možné vodíkové plazma udržet uvnitř reaktoru je nutné využít supravodivé magnety, které jsou schopné generovat dostatečně silné magnetické pole. To bude zabraňovat kontaktu plazmatu s povrchem reaktoru.
				\item Očekávaný termín spuštění je v roce 2025, plného výkonu by měl dosáhnout o deset let později.
			\end{itemize}
		\end{column}
		\begin{column}{.35\textwidth}
			\begin{figure}
				\adjincludegraphics[width=\textwidth]{img/U.S._Department_of_Energy.jpg}
				\caption*{Model reaktoru ITERu.\footnote[frame]{Zdroj: \href{https://commons.wikimedia.org/wiki/File:U.S._Department_of_Energy_-_Science_-_425_003_001_(9786811206).jpg}{U.S. Department of Energy/Commons}}}
			\end{figure}
		\end{column}
	\end{columns}
	\vfill
}

\frame{
	\frametitle{}
	\vfill
	\begin{figure}
		\adjincludegraphics[height=.7\textheight]{img/drone_nov_2024_12.jpg}
		\caption*{Pohled na staveniště ITERu v říjnu 2024.\footnote[frame]{Zdroj: \href{https://www.iter.org/iter-image-galleries/aerial}{ITER}}}
	\end{figure}
	\vfill
}

\section{Vodíkové hospodářství}
\frame{
	\vfill
	\begin{itemize}
		\item Celosvětová spotřeba energie neustále narůstá.
		\item V roce 2019 dosáhla celková světová spotřeba elektřiny 22~848~TWh, což je o 1,7 \% více než v roce 2018.
	\end{itemize}
	\begin{figure}
		\adjincludegraphics[width=.8\textwidth]{img/IEA.png}
		\caption*{Celosvětová spotřeba elektrické energie.\footnote[frame]{Zdroj: \href{https://www.iea.org/reports/electricity-information-overview/electricity-consumption}{IEA}}}
	\end{figure}
	\vfill
}

\frame{
	\vfill
	\begin{itemize}
		\item V roce 2022 bylo v ČR vyrobeno téměř 79 TWh elektrické energie.\footnote[frame]{\href{https://eru.gov.cz/rocni-zprava-o-provozu-elektrizacni-soustavy-cr-pro-rok-2022}{Roční zpráva o provozu elektrizační soustavy ČR pro rok 2022}}
	\end{itemize}
	\begin{center}
		\begin{tabular}{|l|l|r@{,}l|}
			\hline
			\textbf{Zdroj} & \textbf{Vyrobeno [GWh]} &
			\multicolumn{2}{c|}{\textbf{Zastoupení [\%]}} \\\hline
			Jaderné & 29 311 & 37 & 22 \\\hline
			Parní & 37 288 & 47 & 35 \\\hline
			Paroplynové & 2 499 & 3 & 17 \\\hline
			Plynové a spalovací & 3 683 & 4 & 68 \\\hline
			Vodní & 2 077 & 2 & 64 \\\hline
			Přečerpávací & 977 & 1 & 24 \\\hline
			Větrné & 633 & 0 & 80 \\\hline
			Fotovoltaické & 2 280 & 2 & 90 \\\hline
			\textbf{Celkem} & \textbf{78 747} & \textbf{100} & \textbf{00} \\\hline
		\end{tabular}
	\end{center}
	\begin{itemize}
		\item Výkon větrných a fotovoltaických elektráren je závislý na počasí a ročním období, proto je při navyšování jejich podílů v energetickém mixu nutné myslet i na ukládání přebytečné energie.
	\end{itemize}
	\vfill
}

\frame{
	\frametitle{}
	\begin{itemize}
		\item Snaha o snížení množství uhlíku v ekonomice.\footnote[frame]{\href{https://www.youtube.com/watch?v=spVuaexIO5k}{Vodík - palivo pro udržitelnou energetiku}}
		\item Zásoby vodíku na Zemi jsou prakticky nevyčerpatelné.
		\item Vodík se následně přeměňuje na ekologicky nezávadnou(?) vodu.
		\item I když se už vodík v praxi využívá, je stále spousta problémů nevyřešená.
	\end{itemize}
	\begin{figure}
		\adjincludegraphics[height=.48\textheight]{img/Hydrogen.economy.sys_integration_circle.jpg}
		\caption*{Vodíkové hospodářství.\footnote[frame]{Zdroj: \href{https://commons.wikimedia.org/wiki/File:Hydrogen.economy.sys_integration_circle.jpg}{Mion/Commons}}}
	\end{figure}
}

\frame{
	\vfill
	\begin{figure}
		\adjincludegraphics[width=\textwidth]{img/energy-density.png}
		\caption*{Energetická hustota vybraných látek.}
	\end{figure}
	\vfill
}

\subsection{Barvy vodíku}
\frame{
	\frametitle{}
	\begin{itemize}
		\item \textit{Šedý vodík} -- nejběžnější a nejlevnější vodík, získává se rozkladem zemního plynu, zároveň vzniká velké množství \ce{CO2}.
		\begin{itemize}
			\item \ce{CH4 + H2O -> CO + 3 H2}
			\item \ce{CO + H2O -> CO2 + H2}
		\end{itemize}
		\item \textit{Modrý vodík} -- stejný jako šedý, ale \ce{CO2} je zachycován a ukládán.
		\item \textit{Černý a hnědý vodík} -- vyrábí se zplyňováním uhlí nebo biomasy.
		\begin{itemize}
			\item \ce{C + 2 H2O -> CO2 + 2 H2}
		\end{itemize}
		\item \textit{Zelený vodík} -- vyrábí se elektrolyticky, s využitím čisté energie, tzn. energie generované obnovitelnými zdroji -- solárními panely, větrnými elektrárnami, atd.
		\item \textit{Žlutý vodík} -- zelený vodík, zdrojem energie je slunce.
		\item \textit{Růžový vodík} -- stejný jako zelený, ale energie pochází z jaderných elektráren.
		\item \textit{Bílý vodík} -- získává se z geologických ložisek vodíku.
		\item \textit{Tyrkysový vodík} -- získává se pyrolýzou methanu, při které nevznikají žádné uhlíkové emise.
	\end{itemize}
}

\subsection{Výroba vodíku}
\frame{
	\frametitle{}
	\begin{columns}
		\begin{column}{.65\textwidth}
			\textit{Elektrolýza vody}
			\begin{itemize}
				\item \ce{2 H2O -> 2 H2 + O2}
				\item Čistá voda obsahuje velmi málo iontů (vodivost 0,055 $\mu$S.cm$^{-1}$).
				\item \ce{2 H2O <=> H3O+ + OH-}
				\item K$_w$ = 1,0 $\times$ 10$^{-14}$
				\item Aby mohla elektrolýza probíhat je nutné přidat vhodný elektrolyt.
				\item Minimální napětí je 1,23 V.
				\item \ce{2 H2O -> O2 + 4 H+}, E$^0$ = +1,23 V
				\item \ce{2 H+ -> H2}, E$^0$ = 0,00 V
			\end{itemize}
		\end{column}

		\begin{column}{.4\textwidth}
			\begin{figure}
				\adjincludegraphics[width=.85\textwidth]{img/Hofmann_voltameter.png}
				\caption*{Elektrolýza vody.\footnote[frame]{Zdroj: \href{https://commons.wikimedia.org/wiki/File:Hofmann_voltameter_it.svg}{HeNRyKus/Commons}}}
			\end{figure}
		\end{column}
	\end{columns}
}

\frame{
	\frametitle{}
	\textit{Pyrolýza methanu}\footnote[frame]{\href{https://doi.org/10.1021/acs.iecr.1c01679}{Methane Pyrolysis for Zero-Emission Hydrogen Production: A Potential Bridge Technology from Fossil Fuels to a Renewable and Sustainable Hydrogen Economy}}

	\begin{itemize}
		\item Jeden z možných mechanismů produkce vodíku bez emisí \ce{CO2}.
		\item Z ekologického hlediska to není nejoptimálnější metoda, protože je závislá na zemním plynu.
		\item Jako katalyzátory se využívají kovy (Ni, Co, Fe) nebo uhlík.
	\end{itemize}

	\begin{align*}
		\ce{2 CH4 &-> CH3-CH3 + H2} \\
		\ce{CH3-CH3 &-> CH2=CH2 + H2} \\
		\ce{CH2=CH2 &-> CH\bond{3}CH + H2} \\
		\ce{CH2=CH2 + CH4 &-> CH3-CH\bond{2}CH2 + H2} \\
		\ce{CH\bond{3}CH + CH4 &-> CH3-C\bond{3}CH + H2} \\
		\ce{CH3-CH\bond{2}CH2 &-> CH2\bond{2}C\bond{2}CH2 + H2} \\
		\ce{CH3-CH\bond{2}CH2 + CH4 &-> CH3-CH2-CH=CH2 + H2} \\
	\end{align*}
}

\subsection{Skladování vodíku}
\frame{
	\frametitle{}
	\begin{itemize}
		\item Vodík lze skladovat v čistém stavu nebo jako vázaný ve sloučeninách.
		\item Plynný vodík je možné skladovat pod nízkým i vysokým tlakem (30--70 MPa).
		\begin{itemize}
			\item Ke skladování lze využít zásobníky pro zemní plyn
			\item Pro skladování velkých množství lze využít podzemní jeskyně nebo staré doly
		\end{itemize}
		\item Kapalný vodík vyžaduje velmi nízké teploty, jeho teplota varu je $-$252,8~$^\circ$C (20,4~K).
		\item V chemickém stavu je možné vodík ukládat ve formě:
		\begin{itemize}
			\item hydridů kovů (Pd, Pt, $\ldots$)
			\item komplexních hydridů (např. \ce{NaAlH4})
			\item MOFů, příp. COFů
			\item \ce{NH3.BH3}
		\end{itemize}
	\end{itemize}
}

\frame{
	\frametitle{}
	\vfill
	\begin{itemize}
		\item Palladium dokáže absorbovat velká množství vodíku za tvorby nestechiometrického hydridu \ce{PdH_x} ($x < 1$).
		\item Tato schopnost byla poprvé popsána už v roce 1866, kdy Thomas Graham zjistil, že palladium dokáže absorbovat vodík o objemu odpovídající více než 900 násobku jeho vlastního objemu.\footnote[frame]{\href{https://doi.org/10.1098/rspl.1868.0030}{On the relation of hydrogen to palladium}}
		\item Tento proces je reverzibilní, proto je palladium využitelné pro skladování vodíku\footnote[frame]{\href{https://doi.org/10.1021/cr030691s}{Thermal Decomposition of the Non-Interstitial Hydrides for the Storage and Production of Hydrogen}} v rámci vodíkového hospodářství.\footnote[frame]{\href{https://www.youtube.com/watch?v=spVuaexIO5k}{Vodík - palivo pro udržitelnou energetiku}}
		\item Během absorpce vodíku dochází ke změnám fyzikálních vlastností kovu:
		\begin{itemize}
			\item Na rozdíl od jiných kovů neztrácí palladium kujnost.
			\item Vodivost klesá s rostoucí koncentrací vodíku, až do vzniku fáze \ce{PdH_{0.5}}, kdy se hydrid stává polovodičem.
			\item Susceptibilita se silně mění v závislosti na obsahu vodíku.
		\end{itemize}
	\end{itemize}
	\vfill
}

\frame{
	\frametitle{}
	\vfill
	\begin{itemize}
		\item Hydrid také vykazuje supravodivost, kritická teplota je 9~K pro stechiometrii PdH.
		\item U nestechiometrických fází byla také pozorována vysokoteplotní supravodivost (až 273~K)\footnote[frame]{\href{https://doi.org/10.1016/S0921-4534(02)02745-4}{Possibility of high temperature superconducting phases in PdH}} za nízkého tlaku (na rozdíl od hydridů lanthanu).
		\item Schopnost absorpce vodíku (\ce{H2} i \ce{D2}) je silně specifická, palladium nesorbuje ani helium, proto jej lze použít pro průmyslové čištění plynného vodíku.
		\item Pro tyto účely je nutné zabránit tvorbě fáze $\beta$, která způsobuje tvrdnutí materiálu a tím silně omezuje difuzi.
		\item Obě fáze jsou kubické s plošně centrovanou mřížkou.
		\item Při vzniku fáze $\alpha$ dochází jen k malých objemovým změnám, nárůst objemu při vzniku $\beta$ fáze je až 10~\%.
	\end{itemize}
	\vfill
}

\frame{
	\frametitle{}
	\vfill
	\begin{columns}
		\begin{column}{.7\textwidth}
			\begin{itemize}
				\item Jako další materiály pro skladování vodíku jsou perspektivní např. \textit{grafen} a~\textit{MOFy}.
				\item MOF (Metal--Organic Framework) -- anorganicko--organické hybridní materiály s~porézní strukturou.
				\item Jsou tvořeny kovovými ionty propojenými organickými linkery, např. komplexy zinečnatých iontů s kyselinou tereftalovou.
				\item Jejich měrný povrch může být vyšší než 1000~m$^2$.g$^{-1}$.\footnote[frame]{\href{https://doi.org/10.1021/cr200274s}{Hydrogen Storage in Metal–Organic Frameworks}}
			\end{itemize}

			\begin{center}
				\chemfig{HOOC-*6(=-=(-COOH)-=-)}
			\end{center}
		\end{column}
		\begin{column}{.4\textwidth}
			\begin{figure}
				\adjincludegraphics[width=\textwidth]{img/OrthographicView_M-OH-chain.png}
				\caption*{Krystalová struktura MOFu DUT-5.\footnote[frame]{Zdroj: \href{https://commons.wikimedia.org/wiki/File:DUT-5_OrthographicView_M-OH-chain.png}{Canucksplayer/Commons}}}
			\end{figure}
		\end{column}
	\end{columns}
	\vfill
}

\frame{
	\frametitle{}
	\vfill
	\begin{figure}
		\adjincludegraphics[width=.9\textwidth]{img/41467_2020_17755_Fig6_HTML.png}
		\caption*{Struktury MOFů.\footnote[frame]{Zdroj: \href{https://doi.org/10.1038/s41467-020-17755-8}{Understanding the diversity of the metal-organic framework ecosystem}}}
	\end{figure}
	\vfill
}

\frame{
	\frametitle{}
	\vfill
	\begin{figure}
		\adjincludegraphics[width=\textwidth]{img/Isoreticular_metal-organic_frameworks_of_the_IRMOF_family.png}
		\caption*{Struktury MOFů.\footnote[frame]{Zdroj: \href{https://commons.wikimedia.org/wiki/File:Isoreticular_metal-organic_frameworks_of_the_IRMOF_family.png}{François-Xavier Coudert/Commons}}}
	\end{figure}
	\vfill
}

\frame{
	\frametitle{}
	\vfill
	\begin{figure}
		\adjincludegraphics[width=.75\textwidth]{img/Hydrogen-Storage_Compound.jpg}
		\caption*{Struktura MOF-74\footnote[frame]{\href{https://doi.org/10.1039/D0CE01870H}{MOF-74-type frameworks: tunable pore environment and functionality through metal and ligand modification}}, dokáže absorbovat methan i vodík.\footnote[frame]{Zdroj: \href{https://commons.wikimedia.org/wiki/File:Hydrogen-Storage_Compound_(5888008691).jpg}{NIST/Commons}}}
	\end{figure}
	\vfill
}

\frame{
	\frametitle{}
	\vfill
	\begin{columns}
		\begin{column}{.5\textwidth}
			\begin{itemize}
				\item \textbf{Grafen} -- monovrstva tvořená uhlíky v hybridizaci sp$^2$.
				\item Poprvé byl připraven v roce 2004 exfoliací grafitu pomocí lepící pásky.\footnote[frame]{\href{https://www.idnes.cz/zpravy/zahranicni/nobelovu-cenu-za-fyziku-dostali-vedci-za-vyzkum-supertenkeho-uhliku.A101005_120031_zahranicni_aha}{Nobelovu cenu za fyziku dostali vědci za výzkum supertenkého uhlíku}} V roce 2010 byla za tento objev udělena Nobelova cena za fyziku.\footnote[frame]{\href{https://www.nobelprize.org/prizes/physics/2010/press-release/}{The Nobel Prize in Physics 2010}}
				\item Grafen se vodíkem hydrogenuje na grafan. 
				\item Následným zahřátím na 450~$^\circ$C je možné vodík uvolnit.
			\end{itemize}
		\end{column}
		\begin{column}{.5\textwidth}
			\begin{figure}
				\adjincludegraphics[width=0.8\textwidth]{img/Graphit_gitter.png}
				\caption*{Struktura grafitu.\footnote[frame]{Zdroj: \href{https://commons.wikimedia.org/wiki/File:Graphit_gitter.png}{Anton/Commons}}}
			\end{figure}
		\end{column}
	\end{columns}
	\vfill
}

\frame{
	\frametitle{}
	\vfill
	\begin{figure}
		\adjincludegraphics[height=.7\textheight]{img/Graphane.png}
		\caption*{Struktura grafanu.\footnote[frame]{Zdroj: \href{https://commons.wikimedia.org/wiki/File:Graphane.png}{Edgar181/Commons}}}
	\end{figure}
	\vfill
}

\frame{
	\frametitle{}
	\textbf{Využití vodíku}
	\begin{itemize}
		\item Spalování vodíku s kyslíkem je technicky obtížně proveditelné, proto se příliš nevyužívá.
		\item Častější je využití přeměny vodíku v elektrochemických palivových článcích.
		\item Známe mnoho různých typů článků, liší se jak provedením elektrod, tak i samotným mechanismem elektrochemické reakce.
	\end{itemize}
	\begin{figure}
		\adjincludegraphics[height=.37\textheight]{img/Fuel_cell_PEMFC.png}
		\caption*{Schéma palivového článku.\footnote[frame]{Zdroj: \href{https://commons.wikimedia.org/wiki/File:Fuel_cell_PEMFC.svg}{Nécropotame/Commons}}}
	\end{figure}
}

\frame{
	\frametitle{}
	\begin{figure}
		\adjincludegraphics[height=.7\textheight]{img/PEM_fuelcell.png}
		\caption*{Schéma vodíkového článku.\footnote[frame]{Zdroj: \href{https://commons.wikimedia.org/wiki/File:PEM_fuelcell.svg}{Jafet/Commons}}}
	\end{figure}
}

\frame{
	\frametitle{}
	\begin{itemize}
		\item První vodíkový automobil byl v provozu již v roce 1806.\footnote[frame]{\href{https://web.itu.edu.tr/celikmuhamm/bil103/company/webpage.html}{History of Hydrogen Cars}}
		\item Současné vodíkové motory využívají jak spalování vodíku, tak i~palivové články.
		\item V současnosti se intenzivně řeší přechod automobilové dopravy z~fosilních paliv na elektřinu nebo vodík.
	\end{itemize}
	\begin{columns}
		\begin{column}{.5\textwidth}
			\begin{figure}
				\adjincludegraphics[height=.3\textheight]{img/Rivaz_Engine.jpg}
				\caption*{Vodíkový motor z roku 1806.\footnote[frame]{Zdroj: \href{https://commons.wikimedia.org/wiki/File:Rivaz_Engine.jpg}{Commons}}}
			\end{figure}
		\end{column}
		\begin{column}{.5\textwidth}
			\begin{figure}
				\adjincludegraphics[height=.3\textheight]{img/Mazda_RX-8_hydrogen.jpg}
				\caption*{Mazda RX-8 Hydrogen.\footnote[frame]{Zdroj: \href{https://commons.wikimedia.org/wiki/File:Mazda_RX-8_hydrogen_--_2011_DC.jpg}{IFCAR/Commons}}}
			\end{figure}
		\end{column}
	\end{columns}
}

\section{Závěr}
\frame{
	\vfill
	\centering \Huge
	\textbf{Děkuji za pozornost} \\[2ex]

	\large
	Zdeněk Moravec\\
	\href{https://is.muni.cz/www/moravec/}{is.muni.cz/www/moravec/}\\
	hugo@chemi.muni.cz
	\vfill
}

\end{document}